\documentclass[12pt]{beamer}
\usetheme{default}
\usecolortheme{crane}
%\usetheme{Madrid}

\usepackage[utf8x]{inputenc}
\usepackage[T1]{fontenc}
\usepackage[slovak]{babel}
\usepackage{ucs}
\usepackage{amsmath}
\usepackage{graphicx}
\usepackage{array}
\usepackage{amsmath, amssymb}
%\usepackage[inline]{asymptote}

%\setbeamersize{text margin left=1pt,text margin right=1pt}
\setbeamertemplate{footline}[frame number]
\beamertemplatenavigationsymbolsempty

\let\o=\vee
\let\a=\wedge
\let\bigo=\bigvee
\let\biga=\bigwedge

\title{Databázové systémy}
\author{Ján Mazák}
\institute{FMFI UK Bratislava}
\date{}
%\date % (optional)
%{23. 9. 2019}

% database-related stuff
\DeclareMathOperator{\join}{\bowtie}
\DeclareMathOperator{\antijoin}{\rhd}

\DeclareMathOperator{\COUNT}{\textrm{COUNT}}
\DeclareMathOperator{\SUM}{\textrm{SUM}}
\DeclareMathOperator{\MAX}{\textrm{MAX}}


\DeclareMathOperator{\osoba}{osoba}
\DeclareMathOperator{\firma}{firma}
\DeclareMathOperator{\vlastni}{vlastni}
\DeclareMathOperator{\ponuka}{ponuka}
\DeclareMathOperator{\chce}{chce}
\DeclareMathOperator{\lubi}{lubi}
\DeclareMathOperator{\capuje}{capuje}
\DeclareMathOperator{\navstivil}{navstivil}
\DeclareMathOperator{\vypil}{vypil}
\DeclareMathOperator{\answer}{answer}


\begin{document}

\frame{\titlepage}

\begin{frame}
Databáza
\begin{itemize}
\item kolekcia údajov
\item so štruktúrou
\end{itemize}
\vskip 1cm
\pause

DBMS (database management system), voľne tiež databáza
\begin{itemize}
\item obsahuje veci spoločné pre jednotlivé databázy
\item pracuje nad dátami abstraktne, používateľ (tvorca konkrétnej databázy) definuje všetky vzťahy medzi dátami (napr. konzistentnosť)
\end{itemize}
\end{frame}

\begin{frame}
Požiadavky na DBMS
\begin{itemize}
\item trvalé uchovanie (persistency)
\item bezpečnosť (na úrovni hw, sw, používateľov)
\item paralelizmus (veľa používateľov zároveň)
\item pohodlnosť (vysokoúrovňové deklaratívne jazyky)
\item efektívnosť (tisíce dotazov za sekundu)
\end{itemize}
\pause
Aplikácie pracujúce s extrémnymi objemami dát niekedy nevyužívajú DBMS.
\end{frame}

\begin{frame}
Čo všetko súvisí s DBMS?
\begin{itemize}
\item fyzické umiestnenie dát
\item sieťové pripojenia (klienti, distribuovanosť databázy)
\item paralelizmus (veľa používateľov robiacich operácie nad tými istými dátami)
\item optimalizácia dotazov (algoritmická zložitosť)
\end{itemize}
\pause
\end{frame}

\begin{frame}
Dátový model
\begin{itemize}
\item relačný (tabuľky)
\item entitno-relačný
\item objektový
\item hierarchický (strom)
\item dokumentový (XML, json)
\item graf
\item key-value store
\end{itemize}
SQL --- relačný model\\
NoSQL --- key-value store, graph, document...
\end{frame}

\begin{frame}
Databázové jazyky
\begin{itemize}
\item Data Definition Language (DDL)
\item Data Manipulation Language (DML)
\end{itemize}
\vskip 1 cm
\pause
Prístup k dátam z programovacích jazykov:
\begin{itemize}
\item priamo (cez DML)
\item s jednoduchou nadstavbou (prepared statement atď.)
\item ORM (object-relational mapper) --- pracuje sa s objektmi, DML sa automaticky generuje v pozadí 
\end{itemize}
\end{frame}

\begin{frame}
Dotazovacie jazyky
\begin{itemize}
\item chceme, aby aplikácie boli nezávislé od reprezentácie dát
\item optimalizácia na úrovni DBMS, nie u klienta
\pause
\item deklaračné jazyky (len čo chceme, nie ako to vypočítať): relačný kalkul (prvorádové matematické formuly), SQL, Datalog
\pause
\item interné db jazyky (procedurálne zachytávajú postup výpočtu): relačná algebra, fyzické operátory
\end{itemize}
\end{frame}

\begin{frame}
\frametitle{História}
\begin{itemize}
\item 1970: relačný model
\item 1986: prvý štandard SQL
\item od cca 2010: NoSQL, Spark...
\pause
\item reálne systémy nedržia tempo s teóriou ani štandardmi, napr. rekurzia je v najväčších DBMS implementovaná ani 10 rokov, kým štandard je z 1999
\end{itemize}
\end{frame}

\begin{frame}
\frametitle{Tento predmet}
\begin{itemize}
\item zameraný najmä na praktické aspekty práce s relačnými DBMS
\item dotazy
\item navrhovanie databázových schém v relačnom modeli
\item transakcie
\item efektivita (význam indexov, vkladanie veľkých objemov dát)
\pause
\item ale aj prehľad o teoretických aspektoch (vyjadrovacia sila dotazovacích jazykov, zložitosť vybraných algoritmov)
\end{itemize}
\end{frame}

\begin{frame}
\frametitle{Hodnotenie}
\begin{itemize}
\item účasť na prednáškach ani cvičeniach nie je povinná
\item z každého cvičenia však treba spraviť aspoň časť\\(10 x 2 b, treba min. 15)
\item tri domáce úlohy (3 x 20 b)
\item za semester 80 bodov
\pause
\item ústna skúška za 20 bodov, treba min. 10
\end{itemize}
\end{frame}

\begin{frame}
\frametitle{Ukážka relačného kalkulu}
Alkoholy, ktoré ľúbi každý pijan (ktorý niečo ľúbi), čapujú ich všade (kde niečo čapujú) a niekto ich už pil.
{\tiny
\begin{align*}
\Big\{A\mid (\exists I\,\exists M\vypil(I, A, M))&\land \lnot \Big[\exists P (\exists A_2\lubi(P, A_2))\land\lnot\lubi(P, A))\Big)\Big]\\
&\land \lnot\Big[\exists K \Big((\exists A_2 \capuje(K, A_2))\land \lnot\capuje(K, A)\Big)\Big]\Big\}
\end{align*}
}
\end{frame}

\begin{frame}
\frametitle{Ukážka datalogu}
Alkoholy, ktoré ľúbi každý pijan (ktorý niečo ľúbi), čapujú ich všade (kde niečo čapujú) a niekto ich už pil.
{\tiny
\begin{align*}
\operatorname{nelubenyNiekym}(A) &\leftarrow \vypil(\_, A, \_), \lubi(P, \_), \lnot\lubi(P, A).\\
\operatorname{necapovanyNiekde}(A) &\leftarrow \vypil(\_, A, \_), \capuje(K, \_), \lnot\capuje(K, A).\\
\answer(A) &\leftarrow \vypil(\_, A, \_), \lnot\operatorname{nelubenyNiekym}(A), \lnot\operatorname{necapovanyNiekde}(A).\\
\end{align*}
}
\end{frame}

\begin{frame}
\frametitle{Ukážka relačnej algebry}
Alkoholy, ktoré ľúbi každý pijan (ktorý niečo ľúbi), čapujú ich všade (kde niečo čapujú) a niekto ich už pil.
{\tiny
\begin{align*}
\operatorname{nelubenyNiekym} & = (\pi_A(\vypil)\times \pi_P(\lubi))\antijoin \lubi\\
\operatorname{necapovanyNiekde} & = (\pi_A(\vypil)\times \pi_K(\capuje))\antijoin \capuje\\
\answer & = ((\pi_A(\vypil) \antijoin \operatorname{nelubenyNiekym})\antijoin \operatorname{necapovanyNiekde}\\
\end{align*}
}
\end{frame}

\begin{frame}[fragile]
\frametitle{Ukážka SQL}
Alkoholy, ktoré ľúbi každý pijan (ktorý niečo ľúbi), čapujú ich všade (kde niečo čapujú) a niekto ich už pil.
{\tiny
\begin{verbatim}
    SELECT DISTINCT v.A
    FROM vypil v
    WHERE NOT EXISTS (SELECT 1 FROM lubi l
                    WHERE NOT EXISTS (SELECT 1 FROM lubi l2
                                        WHERE l2.P = l.P AND l2.A = v.A))
        AND NOT EXISTS (SELECT 1 FROM capuje c
                        WHERE NOT EXISTS (SELECT 1 FROM capuje c2
                                        WHERE c2.K = c.K AND c2.A = v.A))
\end{verbatim}
}
\end{frame}



\end{document}




