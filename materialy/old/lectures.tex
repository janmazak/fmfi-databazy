\documentclass[10pt, a4paper]{article}

% DRAFT
%\usepackage[light,first,bottomafter]{draftcopy}

%\usepackage{changebar}

% slovencina
\usepackage[activeacute, slovak, english]{babel}
%\noextrasslovak
\usepackage[utf8]{inputenc}
\usepackage[T1]{fontenc}

% rozmery stranky
%\usepackage{a4wide}
\addtolength{\voffset}{-3cm}
\addtolength{\hoffset}{-1.5cm}
\addtolength{\textwidth}{3 cm}
\addtolength{\textheight}{5 cm}
\usepackage{array}
\usepackage{mdwlist}

% fonty pre text a matematiku
%\usepackage{newcent}
%\usepackage{euler}
\usepackage{amsmath}


% AMS-TeX
\usepackage{amsmath}
\usepackage{amsfonts}
\usepackage{amssymb}
\usepackage{multicol}
\usepackage{amsthm}

\def\ans#1{\big[\hskip 2mm {#1}\hskip 2mm\big]}
\def\N{\mathbb N}
\def\Z{\mathbb Z}
\def\Q{\mathbb Q}
\def\R{\mathbb R}
\def\C{\mathbb C}

\begin{document}
\selectlanguage{slovak}

\addtolength{\parskip}{0.5\baselineskip}

\pagestyle{empty}


\section{Relačný kalkul}

Všetky dotazy so všeob. kvantifikátorom následne prepísať cez existenčný.

\begin{enumerate}
\item 
$EDB = \{osoba(A), pozna(Kto, Koho)\}$
\begin{itemize}
    \item osoby, ktoré poznajú sysľa
    \item osoby, ktoré nepoznajú nikoho (žiadne iné osoby)
    \item osoby, ktoré majú aspoň dvoch známych (osoby)
    \item osoby, ktoré pozná presne jedna osoba
    \item osoby, ktoré poznajú iba Jožka
    \item osoby, ktoré majú všetky vzťahy symetrické
    \item osoby, ktoré medzi Ferovými známymi poznajú aspoň jedného, ale nie všetkých
    \item osoby, ktoré poznajú všetkých známych svojich známych
    \item osoby, ktoré poznajú aspoň jedného známeho každého svojho známeho
\end{itemize}

\item 
$EDB = \{integer(X), multiply(X, Y, Result)\}$
\begin{itemize}
    \item hodnota $5\cdot 4$
    \item množina párnych čísel
    \item množina nepárnych čísel
    \item dvojice nesúdeliteľných čísel
\end{itemize}

\end{enumerate}


\section{Relačný kalkul bez všeob. kvantifikátora, datalog}

% \item {\bf Hra telefón.} Rozdeliť sa do trojíc A, B, C; A si pozrie slovný dotaz napísaný zo zadnej strany na tabuli a prepíše ho do formuly, tú ukáže B a potom ju B prepíše slovami,
% výsledok ukáže C a ten slová prepíše do výslednej formuly na tabuľu k zadaniu; spolu zhodnotíme, či tie formuly sú správne (mnoho je zle už pre syntaktické chyby, asi tretina bola správna).
% 
% Paralelne si na druhej tabuli pozrie B slovný dotaz, prepíše ho do formuly pre C, ten do slov pre A a ten píše formulu na tabuľu.
% Na začiatku C len sedia, dať im úlohu o kopaní jamy (viď nižšie.)

Syntax a sémantika datalogu (implicitné existenčné kvantifikátory). Význam premennej $\_$.

\begin{enumerate}
\item
$EDB = \{blysti(Vec), zlate(Vec)\}$
\begin{itemize}
    \item veci, ktoré sú zlaté, ale neblyštia sa
    \item nie je všetko zlato, čo sa blyští    
\end{itemize}

\item
$EDB=\{hodnotenie(Student, Predmet, Znamka)\}$
\begin{itemize}
    \item študenti, ktorí majú aspoň 3 známky A
    \item študenti, ktorí boli hodnotení aspoň raz, ale nemajú Fx
    \item predmety, z ktorých bol každý študent hodnotený viac ako raz (a aspoň jeden študent bol hodnotený)
\end{itemize}

\item
$EDB = \{clovek(Meno), vstupil(Id\_vstupu, Meno, Rieka)\}$
\begin{itemize}
    \item ľudia, ktorí nevstúpili dvakrát do tej istej rieky
    \item rieky, kam vstúpil každý, kto do nejakej rieky vstúpil (a vstúpil do nich aspoň niekto)
    \item ľudia, ktorí vstúpili do každej rieky presne dvakrát
\end{itemize}

\item
$EDB = \{kope(Kto, Komu, Jama), padol(Meno, Jama)\}$
\begin{itemize}
    \item všetci, čo druhému jamu kopú, ale sami do nej padli
    \item tí, čo žiadnu jamu nekopú, ale do nejakej padli
    \item tí, čo padli do každej jamy, ktorú kopali
\end{itemize}

\item
$EDB = \{politik(Meno), cestny(Meno), podplatil(Kto, Koho, Uplatok)\}$
\begin{itemize}
    \item všetci politici sú nečestní    
    \item niektorí nečestní sú politikmi
    \item politici, ktorí nikdy neprijali uplatok
    \item politici, ktorí síce nie sú čestní, ale prijímajú len úplatky do 100 
\end{itemize}

\begin{enumerate}
\item
$EDB = \{kozmonaut(Meno, Vek, Stat, Lod)\}$
\begin{itemize}
    \item lode, ktoré používal Armstrong alebo Gagarin
    \item mená a veky kozmonautov, ktorí použili aspoň dve lode
    \item štáty, ktoré majú práve jedného kozmonauta
\end{itemize}

\end{enumerate}



\section{Datalog}

\item 
$EDB = \{part(Name), component(Item, Subitem, AttachmentType\}$
\begin{itemize}
    \item súčiastky, ktoré sú zložené z aspoň dvoch komponentov
    \item atomické súčiastky (z ničoho sa neskladajú)
    \item všetky komponenty motora (či už atomické alebo nie)
    \item atomické súčiastky potrebné na zloženie televízora
    \item súčiastky, ktoré sú priskrutkované k iným
\end{itemize}

\item $EDB=\{lubi(Pijan, Alkohol), capuje(Krcma, Alkohol),$\\
\hspace*{1cm} $navstivil(Id, Pijan, Krcma), vypil(Id, Alkohol, Mnozstvo)\}$
\begin{itemize}
    \item krčmy, kde sa čapuje pivo a nič iné
    \item krčmy, kde sa pije pivo a nič iné
    \item pijani, ktorí ľúbia rum a práve jeden iný alkohol
    \item pijani-štamgasti, ktorí doteraz pili v jedinej krčme
    \item pijani, ktorí niekedy niekde vypili viac ako pol litra jedného alkoholu
    \item pijani, ktorí nikdy neodolali rumu (pili ho pri každej návšteve krčmy, v ktorej ho čapujú)
    \item pijani, ktorí ľúbia aspoň jeden taký alkohol, ktorý čapuje každá krčma, v ktorej ten pijan niečo vypil
    \item pijani, ktorí pri niektorej svojej návšteve krčmy vytvorili doteraz platný rekord v pití vodky v danej krčme
    \item pijanov, ktorí vypili len tie alkoholy, ktoré vypil pijan Felix (t.j. hľadaní pijani vypili nejakú neprázdnu podmnožinu alkoholov,
            ktoré vypil pijan Felix; a okrem tých alkoholov nevypili žiadne iné)
    \item alkoholy, pre ktoré platí, že ak ten alkohol niektorý pijan niekedy vypil, tak ho ten pijan vypil pri každej svojej návšteve krčmy
            (vo výsledku majú byť aj alkoholy, ktoré nikto nikdy nevypil)
    \item krčmy, pre ktoré platí: hľadanú krčmu nenavštívil žiaden pijan, ktorý ľúbi všetky alkoholy, ktoré tá krčma čapuje (predpokladajte, že každá krčma čapuje nejaký alkohol) 
    \item všetkých takých pijanov, ktorí neľúbia pivo ani borovičku; a zároveň sa dôsledne vyhýbajú návštevám takým krčiem, v ktorých sa čapuje len pivo alebo borovička;
            a zároveň nikdy pivo ani borovičku nevypili
    \item alkoholy, ktoré ľúbia len tí pijani, ktorí nikdy nenavštívili krčmu Wasa
    \item dvojice [P, K] také, že pijan P pri každej návšteve krčmy K vypil niektorý z alkoholov, ktoré ľúbi
            (pri rôznych návštevách mohol vypiť rôzne obľúbené alkoholy, chceme len dvojice, kde P niekedy navštívil K) 
    \item dvojice [P, A], ktoré hovoria, ktoré alkoholy A pijan P vypil pri každej svojej návšteve krčmy (abstinenti nemajú byť vo výsledku)
    \item dvojice [P, A] také, že pijan P ľúbi alkohol A, a zároveň každá krčma, v ktorej P vypil A, čapuje alkohol A lacnejšie než ktorákoľvek iná krčma, ktorá čapuje A
    \item dvojice [K, A] také, že krčma K čapuje alkohol A, a zároveň každý pijan, ktorý ľúbi alkohol A, ho vypil pri niektorej návšteve krčmy K
    \item dvojice [P, A] také, že pijan P ľúbi alkohol A a ešte také dva ďalšie (navzájom rôzne) alkoholy, že pri každej návšteve krčmy,
            pri ktorej P vypil A, vypil aj niektorý z týchto dvoch ďalších alkoholov
    \item dvojice [A, K] také, že alkohol A čapovaný v krčme K vypil (pri aspoň jednej návšteve) každý pijan, ktorý K niekedy navštívil
    \item dvojice [P, A] také, že pijan P ľúbi alkohol A; a v každej krčme, ktorá čapuje alkohol A, vypil P počas niektorej
            návštevy viacej alkoholu A než ktorýkoľvek iný pijan počas jednej návštevy (teda P je rekordérom v pití A na jedno posedenie v každej krčme, ktorá A čapuje)
    \item pijanov, ktorí každý akt vypitia alkoholu urobili v jednej z krčiem, kde je ten alkohol najlacnejší (abstinenti nemajú byť vo výsledku)    
\end{itemize}

\end{enumerate}



\section{cvičenie: datalog, SQL}

\begin{enumerate}
\item {\bf Rozcvička.} $EDB=\{lubi(Pijan, Alkohol), capuje(Krcma, Alkohol),$\\
\hspace*{1cm} $navstivil(Id, Pijan, Krcma), vypil(Id, Alkohol, Mnozstvo)\}$
\begin{itemize}
    \item datalog
    \item dvojice [K, A] také, že krčma K čapuje alkohol A, a zároveň každý pijan, ktorý ľúbi alkohol A, ho vypil pri niektorej návšteve krčmy K (čiže tú krčmu aj navštívil)
\end{itemize}

\item SQL; $EDB=\{hodnotenie(Student, Predmet, Znamka)\}$
\begin{itemize}
    \item študenti, ktorí dostali aspoň jedno A
    \item študenti, ktorí dostali aspoň jedno A a nedostali Fx
    \item študenti, ktorí dostali A aspoň z troch predmetov [využiť agregáciu]
\end{itemize}

\item úloha z rozcvičky  
\begin{itemize}
    \item zapísať datalog (dobrovoľník nejaký)
    \item strojový preklad z datalogu do SQL vysvetliť (ja)
\end{itemize}

\item SQL
$EDB = \{kope(Kto, Komu, Jama), padol(Meno, Jama)\}$
\begin{itemize}
    \item všetci, čo druhému jamu kopú, ale sami do nej padli
    \item tí, čo žiadnu jamu nekopú, ale do nejakej padli
    \item tí, čo padli do každej jamy, ktorú kopali
\end{itemize}

\end{enumerate}




\section{cvičenie: SQL, relačná algebra}

\begin{enumerate}
\item {\bf Rozcvička.} $EDB=\{lubi(Pijan, Alkohol), capuje(Krcma, Alkohol),$\\
\hspace*{1cm} $navstivil(Id, Pijan, Krcma), vypil(Id, Alkohol, Mnozstvo)\}$
\begin{itemize}
    \item datalog
    \item krčmy K, ktoré navštívil aspoň jeden pijan, ktorý ľúbi všetky alkoholy, ktoré K čapuje (predpokladajte, že každá krčma čapuje nejaký alkohol) 
\end{itemize}

\item
$EDB = \{osoba(A), pozna(Kto, Koho)\}$
\begin{itemize}
    \item osoby, ktoré poznajú sysľa
    \item osoby, ktoré nepoznajú nikoho a nič (nielen osoby)
    \item osoby, ktoré majú aspoň štyri známe osoby
    \item osoby, ktoré pozná presne jedna entita
    \item osoby, ktoré poznajú iba Jožka
    \item osoby, ktoré poznajú všetkých známych svojich známych
\end{itemize}

\item vyriešiť rozcvičku (datalog, SQL, algebra)

Relačná algebra; nad multimnožinami:
\begin{itemize}
\item $\cup$ zjednotenie
\item $\cap$ prienik
\item $-$ rozdiel
\item $\varrho$ premenovanie (stĺpcov)
\item $\times$ karteziánsky súčin (počet stĺpcov vo výsledku je súčtom počtov stĺpcov operandov)
\item $\pi_X(r)$ projekcia; $X$ je množina atribútov, ktoré ostanú, ostatné sa zahodia
\item $\sigma_{c(X)}(r)$ selekcia; vyberie riadky, ktoré spĺňajú podmienku $c(X)$
\item $r_1 \bowtie r_2$ natural join
\item $r_1 \bowtie_c r_2$ theta-join = selekcia aplikovaná na karteziánsky súčin
\item $\delta$ eliminácia duplikátov
\item $\Gamma$ grupovanie a agregácia, napr. $\Gamma_{A, B, AVG(C)}(r)$
\item $T$ triedenie
\item $OUTERJOIN$
\end{itemize}
zápisy: jediný výraz / postupné priradzovanie do pomenovaných relácií / strom výpočtu
\end{enumerate}



\section{cvičenie: Datalog, SQL, relačná algebra}

\begin{enumerate}
\item {\bf Rozcvička.} $EDB=\{lubi(Pijan, Alkohol), capuje(Krcma, Alkohol),$\\
\hspace*{1cm} $navstivil(Id, Pijan, Krcma), vypil(Id, Alkohol, Mnozstvo)\}$
\begin{itemize}
    \item relačná algebra, SQL
    \item alkoholy, ktoré čapuje krčma Stein, ale nikdy ich tam nik nepil
\end{itemize}

\item agregácia pijanov: datalog, SQL
$EDB=\{lubi(Pijan, Alkohol), capuje(Krcma, Alkohol, Cena),$\\
\hspace*{1cm} $navstivil(Id, Pijan, Krcma), vypil(Id, Alkohol, Mnozstvo)\}$
\begin{itemize}
    \item pijani, ktorí ľúbia aspoň 10 rôznych alkoholov
    \item alkoholy, ktoré boli vypité v krčme Stein v celkovom množstve väčšom ako 20
    \item dvojice [A, Suma], ktoré popisujú množstvo alkoholu A vypitého v krčme Carlton (vo výsledku len tie, čo sa niekedy pili)
    \item dvojice [P, Pocet], ktoré hovoria, v koľkých krčmách prepil pijan P aspoň 10 EUR počas niektorej (jednej) návštevy
    \item trojice [P, A, Pocet], ktoré hovoria, pri koľkých návštevách pijan P vypil alkohol A (netreba nájsť trojice s počtom 0)
    \item pijani, ktorí sú v niektorej krčme lokálnymi šampiónmi v pití rumu (t.j. hľadaný pijan v aspoň jednej krčme vypil dokopy viacej rumu než ľubovoľný iný pijan)
    \item dvojice [P, Suma], ktoré hovoria, koľko peňazí pijan P celkovo prepil v krčmách, ktoré čapujú len alkoholy, ktoré P neľúbi (dvojice s nulovou sumou nemajú byť vo výsledku)
    \item [P, nK, nA], kde nK je počet rôznych krčiem, ktoré pijan P navštívil, a nA je počet rôznych alkoholov, ktoré pijan P vypil (nechceme trojice, kde $nK=nA=0$) 
    \item [K, A, m], kde m je celkové množstvo alkoholu A vypitého v krčme K (chceme vo výsledku každú dvojicu K, A, kde K čapuje A)
\end{itemize}

\item agregácia pijanov: datalog, SQL
$EDB=\{lubi(Pijan, Alkohol), capuje(Krcma, Alkohol, Cena),$\\
\hspace*{1cm} $navstivil(Id, Pijan, Krcma), vypil(Id, Alkohol, Mnozstvo)\}$
\begin{itemize}
    \item dvojice [K, A] také, že alkohol A sa v krčme K vypil v celkovom množstve väčšom ako 50
    \item trojice [P, K, Pocet], ktoré hovoria, pri koľkých návštevách pijan P vypil v krčme K aspoň 5 borovičiek na jedno posedenie (trojice s nulovým počtom nás nezaujímajú)
    \item dvojice [A, M] také, že M je mediánom ceny alkoholu A cez všetky krčmy, ktoré alkohol A čapujú
    \item dvojice [K, Suma], ktoré hovoria, koľko peňazí v krčme K celkovo prepili pijani, ktorí tú krčmu navštívili viac než stokrát (dvojice s nulovou sumou nemajú byť vo výsledku)
    \item trojice [P, K, Priemer] ktoré hovoria, koľko peňazí utratil pijan P v priemere pri jednej návšteve krčmy K (trojice s nulovým priemerom nemajú byť vo výsledku)
\end{itemize}

\end{enumerate}


\newpage

\section{cvičenie: Datalog, SQL, relačná algebra}

\begin{enumerate}
\item {\bf Rozcvička.} $EDB=\{capuje(Krcma, Alkohol)$
\begin{itemize}
    \item SQL
    \item dvojice [K, $n$] také, že v krčme K sa čapuje $n$ alkoholov s cenou vyššou ako je priemerná cena (cez všetky krčmy, ktoré ich čapujú)
\end{itemize}

\item agregácia pijanov: datalog, SQL
$EDB=\{lubi(Pijan, Alkohol), capuje(Krcma, Alkohol, Cena),$\\
\hspace*{1cm} $navstivil(Id, Pijan, Krcma), vypil(Id, Alkohol, Mnozstvo)\}$
\begin{itemize}
    \item trojice [P, A, Pocet], ktoré hovoria, pri koľkých návštevách pijan P vypil alkohol A (netreba nájsť trojice s počtom 0)
    \item trojice [P, K, Priemer] ktoré hovoria, koľko peňazí utratil pijan P v priemere pri jednej návšteve krčmy K (trojice s nulovým priemerom nemajú byť vo výsledku)
    \item pijani, ktorí sú v niektorej krčme lokálnymi šampiónmi v pití rumu (t.j. hľadaný pijan v aspoň jednej krčme pil aspoň raz a vypil dokopy aspoň toľko rumu ako ľubovoľný iný pijan)
    \item pijani, ktorí sú v každej krčme lokálnymi šampiónmi v pití rumu (t.j. hľadaný pijan v aspoň jednej krčme pil aspoň raz a vypil dokopy aspoň toľko rumu ako ľubovoľný iný pijan)
    \item dvojice [P, Suma], ktoré hovoria, koľko peňazí pijan P celkovo prepil v krčmách, ktoré čapujú len alkoholy, ktoré P neľúbi (dvojice s nulovou sumou nemajú byť vo výsledku)

    
    \item pijani, ktorí ľúbia aspoň 10 rôznych alkoholov
    \item alkoholy, ktoré boli vypité v krčme Stein v celkovom množstve väčšom ako 20
    \item dvojice [A, Suma], ktoré popisujú množstvo alkoholu A vypitého v krčme Carlton (vo výsledku len tie, čo sa niekedy pili)
    \item dvojice [P, Pocet], ktoré hovoria, v koľkých krčmách prepil pijan P aspoň 10 EUR počas niektorej (jednej) návštevy
    \item{} [P, nK, nA], kde nK je počet rôznych krčiem, ktoré pijan P navštívil, a nA je počet rôznych alkoholov, ktoré pijan P vypil (nechceme trojice, kde $nK=nA=0$) 
    \item{} [K, A, m], kde m je celkové množstvo alkoholu A vypitého v krčme K (chceme vo výsledku každú dvojicu K, A, kde K čapuje A)
\end{itemize}

\item agregácia pijanov: datalog, SQL
$EDB=\{lubi(Pijan, Alkohol), capuje(Krcma, Alkohol, Cena),$\\
\hspace*{1cm} $navstivil(Id, Pijan, Krcma), vypil(Id, Alkohol, Mnozstvo)\}$
\begin{itemize}
    \item dvojice [K, A] také, že alkohol A sa v krčme K vypil v celkovom množstve väčšom ako 50
    \item trojice [P, K, Pocet], ktoré hovoria, pri koľkých návštevách pijan P vypil v krčme K aspoň 5 borovičiek na jedno posedenie (trojice s nulovým počtom nás nezaujímajú)
    \item dvojice [A, M] také, že M je mediánom ceny alkoholu A cez všetky krčmy, ktoré alkohol A čapujú
    \item dvojice [K, Suma], ktoré hovoria, koľko peňazí v krčme K celkovo prepili pijani, ktorí tú krčmu navštívili viac než stokrát (dvojice s nulovou sumou nemajú byť vo výsledku)
\end{itemize}

\item pojmy:
\begin{itemize}
    \item funkčná závislosť $X\to Y$
\end{itemize}


\item nájdite funkčné závislosti v reláciách:\\
    (meno, rodné číslo, ulica, mesto, kraj, PSČ, telefónne číslo)\\
    (moleculeId, x, y, z, vx, vy, vz)

\end{enumerate}


\newpage


\section{cvičenie: funkčné závislosti, kľúče}

\begin{enumerate}
\item {\bf Rozcvička.} $EDB=\{lubi(P, A), capuje(K, A),$\\
\hspace*{1cm} $navstivil(Id, P, K), vypil(Id, A, Mnozstvo)\}$
\begin{itemize}
    \item datalog, SQL
    \item všetky dvojice [K, $R$], kde K je krčma, ktorú niekto navštívil,
        a $R\in [0,1]$ je podiel sklamaných pijanov, čiže podiel počtu pijanov, ktorí K navštívili, ale neľúbia žiaden alkohol, ktorý K čapuje,
        k celkovému počtu pijanov, ktorí K navštívili
\end{itemize}

\item Armstrongove axiómy
\begin{enumerate}
    \item $X\subseteq Y \implies Y\to X$
    \item $X\to Y \implies XZ\to YZ$
    \item $X\to Y \wedge Y\to Z\implies X\to Z$
\end{enumerate}

\item \emph{uzáver $X^+$ množiny atribútov $X$} vzhľadom na množinu funkčných závislostí $F$: množina všetkých atribútov $Y$ takých, že $X\to Y$ sa dá odvodiť z $F$
\begin{itemize}
    \item algoritmus: vezmeme $X^+ := X$ a postupne prechádzame všetky závislosti v $F$;
        ak závislosť odvodí nové atribúty, pridáme ich do $X^+$ a závislosť zahodíme (nič nové sa z nej už nebude dať odvodiť);
        ak sa už nedá použiť žiadna zo zvyšných závislostí, končíme
\end{itemize}

\item Nájdite $\{EF\}^+$ vzhľadom na ${\cal F}=\{BE\to GH, BEG\to F, AD\to C, F\to B, BF\to A\}$.

\item \emph{uzáver ${\cal F}^+$ množiny funkčných závislostí ${\cal F}$} je množina funkčných závislostí, ktoré sú dôsledkom ${\cal F}$, t.j. dajú sa odvodiť pomocou Armstrongových axióm
\begin{itemize}
    \item stačí uvádzať \emph{maximálne} funkčné závislosti, t.j. kde sa nedá vynechať atribút z ľavej strany ani pridať atribút na pravú stranu bez porušenia platnosti
\end{itemize}

\item množina funkčných závislostí ${\cal G}$ \emph{pokrýva} množ. funkčných závislostí ${\cal F}$, ak ${\cal F}^+\subseteq {\cal G}^+$
\begin{itemize}
    \item algoritmus: otestujeme, či každú závislosť v ${\cal F}$ vieme odvodiť z ${\cal G}$
    \item test pre jednu závislosť: overíme, či pravá strana závislosti je podmnožinou uzáveru ľavej strany vzhľadom na ${\cal G}$
\end{itemize}

\item minimálne pokrytie množiny funkčných závislostí ${\cal F}$ je množina kanonických funkčných závislostí ${\cal G}$ taká,
    že ${\cal F}$ a ${\cal G}$ sa navzájom pokrývajú a pri vynechaní ľubovoľnej závislosti z ${\cal G}$
    alebo ľubovoľného atribútu z ľavej strany takej závislosti už ${\cal G}$ prestane byť pokrytím ${\cal F}$ 
\begin{itemize}
    \item \emph{kanonická} funkčná závislosť: na pravej strane práve jeden atribút
    \item minimálnych pokrytí môže existovať viac
    \item algoritmus:
        \begin{enumerate}
            \item rozbitie pravých strán (nahradenie $X\to Y$ množinou $\{X\to A, A\in Y\}$)
            \item vynechanie redundantných atribútov na ľavých stranách (každý atribút z ľavej strany otestovať raz, počítame uzáver ľavej strany s vynechaným atribútom vzhľadom na $F$)
            \item vynechanie redundantných závislostí (vezmeme ľavú stranu závislosti $\alpha$ a počítame jej uzáver vzhľadom na $F-\{\alpha\}$, ak obsahuje pravú stranu, závislosť možno vynechať)
        \end{enumerate}
\end{itemize}

\item Nájdite minimálne pokrytie množiny ${\cal F}=\{BE\to GH, BEG\to F, AD\to C, F\to B, BF\to A\}$ pre reláciu $r(A, B, C, D, E, F, G)$.

\item nech $r$ je relácia nad množinou atribútov $U$; \emph{nadkľúč} je množina atribútov $K$ taká, že $K\to U$; \emph{kľúč} je nadkľúč minimálny vzhľadom na inklúziu
\begin{itemize}
    \item algoritmus hľadania kľúčov zhora nadol
        \begin{enumerate}
            \item vezmeme najprv množinu všetkých atribútov a pokúšame sa z nej postupne odstrániť jednotlivé atribúty
            \item podčiarknutý atribút: nachádza sa v každom kľúči
            \item atribúty, ktoré nie sú na pravej strane žiadnej závislosti, sú automaticky podčiarknuté
            \item atribúty, ktoré nie sú na ľavej strane žiadnej závislosti, ale sú na pravej strane, môžeme úplne vynechať, lebo nebudú v žiadnom kľúči
            \item kreslíme strom (pre atribút $X$, ktorý nie je podčiarknutý: ľavá vetva neobsahuje $X$, pravá obsahuje $X$)
            \item pri prehľadávaní:
                \begin{itemize}
                    \item Pred prehľadávaním stromu vypočítaj uzáver z množiny podčiarknutých atribútov --- ak sú podčiarknuté atribúty nadkľúčom, tak to je kľúč (ukonči vetvu)
                    \item V ľavej vetve over, či uzáver naďalej obsahuje ten atribút, ktorý sa v tej vetve vynecháva (ak nie, ukonči vetvu)
                    \item Po nájdení nejakého kľúča neprehľadávaj podstromy, ktoré ten kľúč obsahujú
                \end{itemize}
        \end{enumerate}
\end{itemize}

\item Nájdite všetky nadkľúče pre relačnú schému $r(A, B, C, D, E, F, G)$,\\
${\cal F}=\{ABCD\to EF, ABE\to FG, ABDG\to CF, G\to BD\}$.

\item Uvažujme relačnú schému $r(A, B, C, D, E)$,\\
${\cal F}=\{A\to CE, ACD\to BE, BC\to D, BE\to AC\}$.
\begin{itemize}
    \item[(a)] Nájdite minimálne pokrytie ${\cal F}$.
    \item[(b)] Nájdite všetky kľúče $r$.
\end{itemize}

\item medián v SQL a datalogu: dvojice [P, $m$], kde $m$ je medián počtu čapovaných alkoholov, ktoré pijan P ľúbi (cez všetky krčmy)
\end{enumerate}


\section{cvičenie: funkčné závislosti, kľúče, dekompozícia do normálnych foriem}

\begin{itemize}
\item {\bf Rozcvička.} Dokážte, že ${\cal F} = \{A\to B, A\to C, A\to D, A\to E, BI \to J\}$ je pokrytím ${\cal G} = \{A\to B C E, A B\to D E, B I\to J\}$.
Rozhodnite, či existuje $\varphi\in {\cal F}$ tak, že ${\cal G}^+\not\subseteq ({\cal F}\setminus\{\varphi\})^+$.

\item
Relačná schéma $(r, F)$ je v \emph{tretej normálnej forme} (3NF),
ak pre každú platnú netriviálnu (takú, že ľavá strana sa nedá
skrátiť) funkčnú závislosť $X\to Y$ platí, že buď $X$ je nadkľúč v $r$
alebo $Y$ je časťou nejakého kľúča v $r$.

\item
Relačná schéma $(r, F)$ je v \emph{Boyce-Coddovej normálnej
forme} (BCNF), ak pre každú platnú netriviálnu funkčnú závislosť
$X\to Y$ platí, že $X$ je nadkľúč v $r$

\item 
Dekompozícia do 3NF zachovávajúca funkčné závislosti

Vstup: Relačná schéma r a minimálne pokrytie závislostí F

Výstup: Relačné schémy bestratovej dekompozície do 3NF, pričom všetky funkčné závislosti ostanú zachované

Dekompozícia do 3NF zachovávajúca funkčné závislosti:

Ak F obsahuje závislosť, ktorá obsahuje všetky atribúty r, potom r je už v 3NF

Inak každej funkčnej závislosti v F zodpovedá jedna relačná
schéma

Ak je niektorá podschéma podmnožinou inej, treba ju vynechať

Ak žiadna z tých relačných podschém neobsahuje kľúč r, treba
ešte pridať schému s nejakým kľúčom. (Testovanie, či podschéma
obsahuje nejaký kľúč r, nevyžaduje výpočet kľúčov. Stačí overiť, či
atribúty niektorej podschémy sú nadkľúčom v r. Ak áno, tak tá
podschéma obsahuje nejaký kľúč r, inak neobsahuje.)

\item
Algoritmus dekompozície relácie R do BCNF (je prirodzené začať
s dobrou 3NF dekompozíciou):

1. Dekomponuj r do 3NF so zachovaním funkčných závislostí.

2. Over, či každá relácia dekompozície je v BCNF. Ak nie, nájdi v
nej funkčnú závislosť $X\to Y$ ktorá porušuje BCNF a
dekomponuj reláciu ri do dvoch relácií, $r_i - Y$ a $XY$.

3. Opakuj overovanie a rozkladanie až kým sú všetky relácie v
BCNF.

\item chase
1. Vyrob maticu S s 1 riadkom pre každú podreláciu $r_1, \dots, r_m$
a s 1 stĺpcom pre každý atribút r

2. Nastav počiatočné hodnoty, $S[i, j] := b_{i,j}$

3. for (i = 1...m)
for (j = 1...n)
if (atribút $A_j$ patrí do $r_i$) then $S[i, j] := a_j$

4. Opakuj kým matica mení:
for all ($X\to Y \in F$)
for (všetky dvojice riadkov s rovnakými symbolmi v tých stĺpcoch, ktoré
zodpovedajú atribútom v X)
Zjednoť riadky v tých 2 riadkoch, pritom preferuj symboly a*.

5. Ak jeden riadok obsahuje len symboly a*, rozklad je bezstratový, inak nie.


\item
Pre danú relačnú schému $r$ a množinu funkčných závislostí ${\cal F}$:
\begin{itemize}
    \item[(a)] Nájdite všetky kľúče $r$.
    \item[(b)] Nájdite minimálne pokrytie ${\cal F}$.
    \item[(c)] Zo zostrojeného minimálneho pokrytia vytvorte dekompozíciu do 3NF zachovávajúcu závislosti.
    \item[(d)] Použite chase a dokážte, že vytvorená dekompozícia sa spája bezstratovo.
    \item[(e)] Rozhodnite, či je vytvorená dekompozícia v BCNF.
\end{itemize}
\begin{enumerate}
    \item TODO preratat alebo vymenit zadanie
    \item $r(A, B, C, D, E, F, G)$, ${\cal F}=\{AB\to C, ACEF\to G, BC\to AD, D\to E, DE\to F, DF\to C, F\to DE, G\to F\}$; [kluce: AB, BC, BD, BF, BG]
    \item $r(A, B, C, D, E, F, G, H)$ s funkčnými závislosťami $BE\to GH, BEG\to FA, D\to C, F\to B, BF\to A$.
    \item $r(A, B, C, D, E, F, G, H)$, ${\cal F}=\{BF\to ACG, C\to AE, AH\to F, AF\to H, BG\to F, E\to G, BCE\to F, GH\to AF, H\to D\}$
\end{enumerate}

% \item domáca úloha na mierne zlepšenie hodnotenia: máme $r(X)$, nájdite medián $X$; nájdite reláciu v BCNF, ktorá obsahuje atribúty A, B také, že r-AB\to A (vid tu podmienku v algoritme)

\end{itemize}

\section{cvičenie: funkčné závislosti, kľúče, dekompozícia do normálnych foriem}

\begin{itemize}

\item mal plachetka
\end{itemize}

\newpage

\section{cvičenie: transakcie}

\begin{itemize}
\item rozcvička: nájdite minimálne pokrytie a zostrojte z neho bezstratovú dekompozíciu do 3NF, ktorá zachováva funkčné závislosti:\\
$r(A, B, C, D, E, F, G, H)$ s funkčnými závislosťami $BE\to GH, BEG\to FA, D\to C, F\to B, BF\to A$.

\item
\begin{itemize}
    \item v rozvrhu sú dve operácie \emph{konfliktné}, ak patria rôznym
    transakciám, ich operandom je rovnaký objekt a aspoň jedna z
    tých operácií je write

    \item dva rozvrhy sú konflikt-ekvivalentné práve vtedy, ak
    pozostávajú z rovnakých operácií a
    relatívne poradie každých dvoch konfliktných operácií je rovnaké
    v oboch históriách

    \item rozvrh je \emph{sériový} práve vtedy, ak je kompletný (t.j.
    každá transakcia v tom rozvrhu končí commitom alebo abortom) a
    pre každú dvojicu transakcií T1, T2 platí, že buď všetky operácie
    T1 v tom rozvrhu predchádzajú operáciám T2 alebo naopak

    \item rozvrh je \emph{konflikt-sériovateľný} práve vtedy, ak jeho
    projekcia na commitované transakcie je konflikt-ekvivalentná
    niektorému sériovému rozvrhu tých commitovaných transakcií

    \item {\sl Rozvrh je konflikt-sériovateľný práve vtedy, ak jeho
    precedenčný graf je acyklický. Topologické
    usporiadanie jeho precedenčného grafu hovorí ktorému
    sériovému rozvrhu je ten rozvrh ekvivalentný.}
\end{itemize}

\item rozhodnite, či sú rozvrhy konflikt-sériovateľné
\begin{enumerate}
    \item r1(X), w1(X), r3(Y), r2(X), r1(Y), w2(Y), w3(X)
    \item r1(X), w1(X), r2(Y), w2(Y), r3(Y), r2(X), r1(Y), w3(X)
\end{enumerate}


\item 
\begin{itemize}
\item \emph{transakcia T2 číta X od
transakcie T1}, ak v tom rozvrhu existuje operácia w1(X) a neskôr
operácia r2(X), pričom T1 je v čase operácie r2(X) stále aktívna
(neukončená)

\item
dva rozvrhy H a H' sú \emph{view-ekvivalentné}, ak (sú
definované nad tými istými transakciami a zároveň)
(1) pre každú dvojicu operácií v H, kde nejaká transakcia T1 číta X
od T2 existuje taká istá dvojica operácií v H', kde T1
tiež číta X od
T2, a zároveň
(2) pre každý dátový objekt X, ak transakcia Ti
je posledná
transakcia ktorá píše do X v H, tak aj v H' je Ti
je posledná
transakcia ktorá píše do X (final write)
(Intuitívne, rozvrhy sú view-ekvivalentné, ak majú rovnaký efekt.)

\item rozvrh je \emph{view-sériovateľný}, ak commitovaná projekcia
každého jeho prefixu je view-ekvivalentná niektorému
sériovému rozvrhu
\end{itemize}

\item  rozhodnite, či nasledujúce rozvrhy sú konflikt-sériovateľný a view-sériovateľný
\begin{enumerate}
    \item r3(b), w3(b), w4(b), r2(b), r1(a), r1(c), w1(a), w1(c), r3(a), w3(c), r2(a), w3(c)
    \item r1(a), r1(b), w1(a), r3(a), r2(b), w3(c), r2(c), w2(b), r1(a), w3(a), w2(c), w2(a)
    \item r3(Z), w3(X), w1(Z), w2(X), r2(Z), r2(Y),w1(X), w3(Z), w1(Y), c1, r3(X), c2, c3 
\end{enumerate}

\item dvojfázové zamykanie: transakcia musí na čítanie i písanie vlastniť zámok; po uvoľnení nejakého zámku už nesmie o iný požiadať; read lock môže mať viac transakcií zároveň, write lock je úplne exkluzívny

\item vytvorte rozvrh pre nasledujúce transakcie, overte, že je konflikt-sériovateľný, a nájdite ekvivalentný sériový rozvrh:
T1: r(X), w(X), r(Y), c
T2: r(Y), w(Y), r(X), c
T3: r(Y), w(X), c

\item 
Uveďte príklad rozvrhu, v ktorom všetky transakcie končia commitom, a ktorý
je konflikt-sériovateľný a nedá sa generovať dvojfázovým zamykaním [r1(X), r2(X), w1(X), c1, c2]


\end{itemize}

\end{document}
















\item $EDB=\{lubi(Pijan, Alkohol), capuje(Krcma, Alkohol),$\\
\hspace*{1cm} $navstivil(Id, Pijan, Krcma), vypil(Id, Alkohol, Mnozstvo)\}$
\begin{itemize}
    \item pijani, ktorí vypili aspoň dva rôzne alkoholy
    \item krčmy, kde sa čapuje pivo a nič iné
    \item krčmy, kde sa pije pivo a nič iné
    \item pijani, ktorí ľúbia len rum
    \item pijani, ktorí ľúbia práve jeden alkohol
    \item pijani-štamgasti, ktorí doteraz navštívili jedinú krčmu
    \item pijani, ktorí niekedy niekde vypili viac ako pol litra jedného alkoholu
    \item pijani, ktorí nikdy neodolali rumu (pili ho pri každej návšteve krčmy, v ktorej ho čapujú)
    \item pijani, ktorí ľúbia aspoň jeden taký alkohol, ktorý čapuje každá krčma, v ktorej ten pijan niečo vypil
    \item pijani, ktorí pri niektorej svojej návšteve krčmy vytvorili doteraz platný rekord v pití vodky v danej krčme
    \item pijanov, ktorí vypili len tie alkoholy, ktoré vypil pijan Felix (t.j. hľadaní pijani vypili nejakú neprázdnu podmnožinu alkoholov,
            ktoré vypil pijan Felix; a okrem tých alkoholov nevypili žiadne iné)
    \item alkoholy, ktoré boli vypité v krčme Stein v celkovom množstve väčšom ako 20
    \item alkoholy, pre ktoré platí, že ak ten alkohol niektorý pijan niekedy vypil, tak ho ten pijan vypil pri každej svojej návšteve krčmy
            (vo výsledku majú byť aj alkoholy, ktoré nikto nikdy nevypil)
    \item krčmy K, ktoré nenavštívil žiaden pijan, ktorý ľúbi všetky alkoholy, ktoré tá krčma čapuje (predpokladajte, že každá krčma čapuje nejaký alkohol) 
    \item všetkých takých pijanov, ktorí neľúbia pivo ani borovičku; a zároveň sa dôsledne vyhýbajú návštevám takým krčiem, v ktorých sa čapuje len pivo alebo borovička;
            a zároveň nikdy pivo ani borovičku nevypili
    \item alkoholy, ktoré ľúbia len tí pijani, ktorí nikdy nenavštívili krčmu Wasa
    \item dvojice [P, K] také, že pijan P pri každej návšteve krčmy K vypil niektorý z alkoholov, ktoré ľúbi
            (pri rôznych návštevách mohol vypiť rôzne obľúbené alkoholy, chceme len dvojice, kde P niekedy navštívil K) 
    \item dvojice [Pijan, Alkohol], ktoré hovoria, ktoré alkoholy ten pijan vypil pri každej svojej návšteve krčmy (abstinenti nemajú byť vo výsledku)
    \item dvojice [P, A] také, že pijan P ľúbi alkohol A, a zároveň každá krčma, v ktorej P vypil A, čapuje alkohol A lacnejšie než ktorákoľvek iná krčma, ktorá čapuje A
    \item dvojice [K, A] také, že krčma K čapuje alkohol A, a zároveň každý pijan, ktorý ľúbi alkohol A, ho vypil pri niektorej návšteve krčmy K
    \item trojice [P, A1, A2] také, že pijan P ľúbi oba rôzne alkoholy A1, A2; a zároveň pri každej návšteve krčmy vypil práve jeden z týchto alkoholov
    \item trojice [P1, P2, K] také, že pijani P1 aj P2 krčmu K niekedy navštívili, ale (nikdy) v K nevypili (žiaden) alkohol, ktorý obaja ľúbia
    \item dvojice [P, A] také, že pijan P ľúbi alkohol A a ešte také dva ďalšie (navzájom rôzne) alkoholy, že pri každej návšteve krčmy,
            pri ktorej P vypil A, vypil aj niektorý z týchto dvoch ďalších alkoholov
    \item dvojice [A, K] také, že alkohol A čapovaný v krčme K vypil (pri aspoň jednej návšteve) každý pijan, ktorý K niekedy navštívil
    \item dvojice [P, A] také, že pijan P ľúbi alkohol A; a v každej krčme, ktorá čapuje alkohol A, vypil P počas niektorej
            návštevy viacej alkoholu A než ktorýkoľvek iný pijan počas jednej návštevy (teda P je rekordérom v pití A na jedno posedenie v každej krčme, ktorá A čapuje)
    \item trojice [A, K1, K2] také, že alkohol A sa čapuje v krčme K1 aj K2; a zároveň každý pijan, ktorý ľúbi alkohol A, vypil A v K1 aj v K2
    \item dvojice pijanov [P1, P2], o ktorých platí, že P1 ľúbi niektorý z alkoholov, ktorý P2 vypil v aspoň troch rôznych krčmách;
            a súčasne P1 nikdy nevypil žiaden z alkoholov, ktoré vypil P2
    \item dvojice [P, K] o ktorých platí, že pijan P navštívil krčmu K práve trikrát, pričom pri každej z tých troch návštev vypil 1 pivo a 1 borovičku a nič iné
    \item trojice [P, A1, A2] také, že pijan P ľúbi aspoň jeden z alkoholov A1 a A2 a pri každej svojej návšteve krčmy Feuerstein oba tie alkoholy (A1 aj A2) vypil
    \item pijanov, ktorí každý akt vypitia alkoholu urobili v jednej z krčiem, kde je ten alkohol najlacnejší (abstinenti nemajú byť vo výsledku)    
\end{itemize}


\item pijani s evidenciou casu navstev
$EDB=\{lubi(Pijan, Alkohol), capuje(Krcma, Alkohol),$\\
\hspace*{1cm} $navstivil(Id, Pijan, Krcma, Od), vypil(Id, Alkohol, Mnozstvo)\}$  (Atribút Od je čas, kedy návšteva začala.)
\begin{itemize}
\item \emph{Držgroš} je pijan, ktorý si pri každej návšteve krčmy všimne a zapamätá ceny
všetkých alkoholov, ktoré tá krčma čapuje. Pri návšteve krčmy je ochotný (ale
nemusí) vypiť len najlacnejší alkohol, ktorý tá krčma čapuje a ktorý on zároveň
ľúbi, a aj to len vtedy, ak zatiaľ nepozná (t.j. predtým nenavštívil) krčmu, ktorá ten
alkohol čapuje lacnejšie. Nájdite všetkých držgrošov.
(Patria medzi nich aj abstinenti. Vo výsledku chceme len pijanov, ktorí aspoň raz navštívili krčmu.) 

\item Pijan je \emph{silne závislý} na niektorom alkohole vtedy, ak ten alkohol konzumuje pri
každej návšteve krčmy, ktorá ten alkohol čapuje; a zároveň platí, že množstvá toho
alkoholu, ktoré pije pri takých návštevách, tvoria s rastúcim časom neklesajúcu
postupnosť. Nájdite všetky dvojice [P, A] také, že pijan P je silne závislý na alkohole A.

\item Pijan je \emph{lojálny ku krčme} K, ak žiaden z alkoholov, ktorý pil v K, už potom nikde inde nepil.
Nájdite všetky dvojice [P, K] také, že pijan P je lojálny ku krčme K.


\item agregácia pijanov
$EDB=\{lubi(Pijan, Alkohol), capuje(Krcma, Alkohol, Cena),$\\
\hspace*{1cm} $navstivil(Id, Pijan, Krcma), vypil(Id, Alkohol, Mnozstvo)\}$
\begin{itemize}
    \item alkoholy, ktoré boli vypité v krčme Stein v celkovom množstve väčšom ako 20
    \item dvojice [P, Pocet], ktoré hovoria, v koľkých krčmách prepil pijan P aspoň 10 EUR počas niektorej (jednej) návštevy
    \item dvojice [K, A] také, že alkohol A sa v krčme K vypil v celkovom množstve väčšom ako 50
    \item trojice [Pijan, Alkohol, Pocet], ktoré o hovoria, pri koľkých návštevách ten pijan ten alkohol vypil (netreba nájsť trojice s počtom 0)
    \item trojice [P, K, Pocet], ktoré hovoria, pri koľkých návštevách pijan P vypil v krčme K aspoň 5 borovičiek na jedno posedenie (trojice s nulovým počtom nás nezaujímajú)
    \item pijani, ktorí sú v niektorej krčme lokálnymi šampiónmi v pití rumu (t.j. hľadaný pijan v aspoň jednej krčme vypil dokopy viacej rumu než ľubovoľný iný pijan)
    \item trojice [K, P1, P2] také, že pijan P1 navštívil krčmu K viackrát než pijan P2 (P2 nemusel krčmu K vôbec navštíviť); a zároveň pijan P1 minul v krčme K viacej
            peňazí než pijan P2 (P2 nemusel v krčme K minúť žiadne peniaze). % TODO doplnit do databazy cenu k capovanemu alkoholu
    \item dvojice [K, A], o ktorých platí, že krčma K čapuje alkohol A a zároveň v krčme K sa alkohol A vypil v menšom celkovom množstve než
            v ktorejkoľvek inej krčme, ktorá čapuje alkohol A (pozor, v krčme K sa alkohol A možno nikdy nepil)
    \item dvojice [A, M] také, že M je mediánom ceny alkoholu A cez všetky krčmy, ktoré alkohol A čapujú
    \item pijani, ktorí ľúbia aspoň 10 rôznych alkoholov
    \item pijanov, pre ktorých platí, že existuje aspoň jeden alkohol, ktorý ten pijan vypil pri každej svojej návšteve krčmy; a zároveň platí, že
            ten alkohol vypil v priemernom množstve aspoň 3 na jednu návštevu
    \item dvojice [Alkohol, Suma], ktoré popisujú množstvo alkoholu vypitého v krčme Carlton (vo výsledku len tie, čo sa niekedy pili)
    \item dvojice [P, Suma], ktoré hovoria, koľko peňazí pijan P celkovo prepil v krčmách, ktoré čapujú len alkoholy, ktoré P neľúbi (dvojice s nulovou sumou nemajú byť vo výsledku)
    \item dvojice [K, Suma], ktoré hovoria, koľko peňazí v krčme K celkovo prepili pijani, ktorí tú krčmu navštívili viac než stokrát (dvojice s nulovou sumou nemajú byť vo výsledku)
    \item trojice [P, K, Priemer] ktoré hovoria, koľko peňazí utratil pijan P v priemere pri jednej návšteve krčmy K (trojice s nulovým priemerom nemajú byť vo výsledku)
\end{itemize}

